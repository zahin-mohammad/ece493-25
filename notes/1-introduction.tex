\documentclass{article}
\usepackage[utf8]{inputenc}
\usepackage{biblatex}
\usepackage{amssymb}
\usepackage{amsmath}
\addbibresource{bib.bib}
\setlength{\parindent}{0em}
\bibliography{bib}


\title{Transformadas de Fourier}
\author{Sara Martins Bonito}
\date{Dezembro de 2019}
\linespread{1.5}
\begin{document}

\maketitle

\section{Introdução}

No seminário do Professor Jorge Drumond Silva\cite{drumond} foram expostas diferentes tipos de equações de onda e dispersão, usadas para modelar os mais diversos fenómenos físicos. Em particular, foi discutida a equação do calor, a qual motiva as séries de Fourier.

Neste trabalho irei fazer a passagem das séries de Fourier estudadas em Análise Complexa e Equações Diferenciais para a transformada de Fourier.

Nas secções seguintes, sempre que for admitida uma função \(f\), esta será integrável em módulo, isto é, \(f\in L^1 \).

\section{Série de Fourier}
Começamos por definir a série de Fourier. Seja \(f:\mathbb{R}\rightarrow\mathbb{R}\) uma função periódica de período \(2L\), integrável em valor absoluto. A série de Fourier de \(f\) é dada por:
\[f(x)\sim\frac{1}{2}a_0+ \displaystyle\sum_{n=1}^{\infty}(a_n\cos{(\frac{nx\pi}{L})}+b_n\sin{(\frac{nx\pi}{L})})\]
Onde \(a_n\) e \(b_n\) para \(\forall n \in \mathbb{N}\) são dados por:
\[a_n=\frac{1}{L}\displaystyle\int_{-L}^{L}f(x)\cos{(\frac{nx\pi}{L})}\quad dx\]
e
\[b_n=\frac{1}{L}\displaystyle\int_{-L}^{L}f(x)\sin{(\frac{nx\pi}{L})}\quad dx\]
Calcula-se \(a_0\) fazendo:
\[a_0=\frac{1}{L}\displaystyle\int_{-L}^{L}f(x)\quad dx\]
\section{Forma Complexa da Série de Fourier}
Sabemos da fórmula de Euler que
\[e^{i\theta}=\cos{\theta}+i\sin{\theta}\]
Como consequência, temos
\[\cos{\theta}=\frac{e^{i\theta}+e^{-i\theta}}{2}\quad{,}\quad \sin{\theta}=\frac{e^{i\theta}-e^{-i\theta}}{2i}\]
Utilizando estas identidades, podemos reescrever a série de Fourier e os integrais que definem os termos da série\cite{djairo}. Os novos termos serão os \(c_n\) obtidos abaixo.
Para \(n\geq0\):
\[c_n=\frac{a_n}{2}-i\frac{b_n}{2}\quad{=}\quad\frac{1}{2L}\displaystyle\int_{-L}^{L}f(x)[\cos{(-\frac{nx\pi}{L})}+i\sin{(-\frac{nx\pi}{L})}]\quad dx\]
Para \(n<0\):
\[c_n=\frac{a_n}{2}+i\frac{b_n}{2}\quad{=}\quad\frac{1}{2L}\displaystyle\int_{-L}^{L}f(x)[\cos{(-\frac{nx\pi}{L})}+i\sin{(-\frac{nx\pi}{L})}]\quad dx\]
Como
\[\cos{(-\frac{nx\pi}{L})}+i\sin{(-\frac{nx\pi}{L})}\quad{=}\quad e^{-i\frac{nx\pi}{L}}\]
Definimos uma única fórmula para os \(c_n\) com \(n\in\mathbb{Z}\), isto é:
\[c_n=\frac{1}{2L}\displaystyle\int_{-L}^{L}f(x)e^{-i\frac{nx\pi}{L}} \quad dx\]
A série de Fourier é agora
\[f\sim \displaystyle\sum_{-\infty}^{+\infty} c_ne^{i\frac{nx\pi}{L}}\]
Repare-se que, apesar da série ter termos complexos, uma função real de variável real apresenta apenas termos reais (graças às propriedades da exponencial complexa).
No entanto, torna-se possível admitir funções \(f:\mathbb{R}\rightarrow\mathbb{C}\) com \(f(x)= u(x)+iv(x)\).

Em análise de Fourier é frequente utilizar-se o símbolo \(\hat{f}(n)\in\mathbb{C}\) para representar os \(c_n\) definidos acima.

Olhemos agora para uma restrição de \(f\) em \([-L,L]\). Podemos fazer uma identificação entre os valores de \(x\in[-L,L]\) e cada \(x'=x+2L\), uma vez que \(f(x)=f(x+2L)\), por definição da função. Obtém-se, assim, uma relação de equivalência \(f(x)\equiv f(x+2L)\) e \(f\) passa a ser \(f:\frac{\mathbb{R}}{2L\mathbb{Z}}\rightarrow\mathbb{C}\). De uma função definida em \(\mathbb{R}\) passamos para uma função definida numa circunferência de perímetro \(2L\).

Geralmente é utilizado o período \(2\pi\), isto é, \(L=\pi\), de onde segue que uma função é tal que \(f: \mathbb{R}\) \(\mod{2\pi}\)\( \rightarrow \mathbb{C}\).  A \(\mathbb{R}\)\(\mod{2\pi}\) também é chamado toro uni-dimensional, \(\mathbb{T}\), cuja notação irei preferir.
Assim, para uma função definida em \([-\pi,\pi]\)
\[f\sim \displaystyle\sum_{n=-\infty}^{\infty}\hat{f}(n)e^{inx}\]
Onde 
\[\hat{f}(n)=\frac{1}{2\pi}\displaystyle\int_{-\pi}^{\pi}f(x)e^{-inx}\quad dx \quad{=}\quad \frac{1}{2\pi}\displaystyle\int_{\mathbb{T}}f(x)e^{-inx}\quad dx\]
Nas próximas secções, as funções utilizadas serão definidas para o intervalo \([-\pi,\pi]\), uma vez que esse intervalo facilitará os cálculos. 
\section{Série de Fourier na circunferência unitária}

Define-se agora um isomorfismo \(\phi\) entre \(\mathbb{T}\) e a circunferência unitária em \(\mathbb{C}\).

Seja \(S^{1}\)=\(\{ z \in\mathbb{C}:|z|=1\}\) a circunferência unitária. \(\{\mathbb{T},+ \}\) e \(\{S^{1},\times\}\) são grupos abelianos, e \(\phi: \mathbb{T}\rightarrow S^{1}\), onde \(\phi(x)=e^{ix}\). 

Usando \(L=\pi\) para cada \(x,y\in[-2\pi,2\pi]\), a sua soma corresponde a adicionar ângulos em \(S^{1}\).

Note-se que definiu-se \(\phi(x)\)=e\(^{inx}\) para n=1. Se \(n\neq1\), temos apenas um homomorfismo, uma vez que a aplicação deixará de ser bijectiva.

Estes homomorfismos são também chamados caracteres, uma vez que são a representação do grupo \{\(\mathbb{T}\),+\} usando funções complexas.

Definimos ainda um produto interno em \(\mathbb{T}\) para duas funções \(f\) e \(g\), de forma a satisfazer \(\langle f,g\rangle\)=\(\overline{\langle g,f\rangle}\) e ter a propriedades de produto interno\cite{maggie}. Então:
\[\langle f,g\rangle=\frac{1}{2\pi}\displaystyle\int_{\mathbb{T}}f(x)\overline{g(x)}\quad dx \]
É fácil ver ainda que para \(n,m\in \mathbb{Z}\):
\[\langle e^{inx},e^{imx}\rangle= \frac{1}{2\pi}\displaystyle\int_{\mathbb{T}}e^{inx}e^{-imx}\quad dx= \frac{1}{2\pi}\displaystyle\int_{\mathbb{T}}e^{i(n-m)x}\quad dx\]
Calculando o integral, vemos que as funções \(e_n(x)=e^{inx}\) formam uma base ortonormada, uma vez que:
\[\frac{1}{2\pi}\displaystyle\int_{\mathbb{T}}e^{i(n-m)x}\quad dx= \begin{cases} 1\quad se\quad m=n\\ 0 \quad se \quad m\neq n\end{cases}\]
Assim, os termos da série de Fourier são
\[\hat{f}(n)=\frac{1}{2\pi}\displaystyle\int_{\mathbb{T}}f(x)e^{-inx}\quad dx= \langle f,e^{inx}\rangle\]
Ao considerarmos os coeficientes da série de Fourier como o produto interno entre \(f\) e uma base ortonormada motiva-se o facto da transformada de Fourier funcionar em \(\mathbb{T}\to\mathbb{Z}\). O mesmo já não acontece quando estendemos a transformada para \(\mathbb{R}\) ou \(\mathbb{R}^{n}\), uma vez que uma qualquer variante de \(e_n(x)=e^{inx}\) não funciona como base nesses dois casos.

\section{A Transformada de Fourier}
Na passagem do toro uni-dimensional {$\mathbb{T}$,+} para {$\mathbb{Z}$,+}, os termos \(\hat{f}(n)\) constituem efectivamente uma transformada de Fourier. Olhando para \(\hat{f}(n)\) não como coeficientes, mas como uma transformação de funções periódicas para sucessões, podemos definir a transformada de Fourier para este caso específico.

\(\textbf{Definição:}\) Dada uma função \(f\) em \{\(\mathbb{T}\),+\}, \(f\in L^1(\mathbb{T})\), definimos a sua transformada de Fourier, \(\mathcal{F}\), como
\[\mathcal{F}(f)(n)=\hat{f}(n)=\displaystyle\int_{\mathbb{T}}f(x)e^{-inx}\quad dx,\quad \hat{f}(n)\in\{\mathbb{Z},+\}\]
A inversa da transformada de Fourier, \(\mathcal{F}^{-1}\) é dada por
\[\mathcal{F}^{-1}(g)(x)=\frac{1}{2\pi}\displaystyle\sum_{-\infty}^{+\infty}\hat{g}(n)e^{inx}\]

\section{Generalização da transformada de Fourier}
\subsection{Para \(\mathbb{R}\)}
A passagem da transformada de Fourier para \(\mathbb{R}\) pode ser feita "passando-a ao limite", uma vez que, de forma não rigorosa, um somatório infinito poderá ser interpretado como um integral impróprio, fazendo \(L\to \infty\). Enquanto \(f\in L^{1}\), podemos fazer a transformada de Fourier na mesma.
Os caracteres são agora contínuos da forma \(e^{i\xi x} :\mathbb{R}^{+}\to\mathbb{T}\).
Seja \(\xi=\frac{n\pi}{L}\). A transformada de Fourier usando \(\xi\) é:
\[\mathcal{F}(f)(x)=\hat{f}(\xi)=\displaystyle\int_{-L}^{L}f(x)e^{-i\xi x}\quad dx\]
Com \(L\to\infty\):
\[\displaystyle\int_{-\infty}^{+\infty}f(x)e^{-i\xi x}\quad dx= \hat{f}(\xi)\]
Prossegue-se da mesma forma para a inversa:
\[\mathcal{F}^{-1}(g)(x)=\frac{1}{2L}\displaystyle\sum_{-\infty}^{+\infty}2L\hat{g}(n)e^{i\frac{n\pi}{L}x}=\frac{1}{2\pi}\displaystyle\int_{-\infty}^{+\infty}\hat{g}(\xi)e^{i\xi x}\quad d\xi\]
Que são as expressões obtidas nas notas do seminário\cite{drumond}, a menos de \(\frac{1}{2\pi}\). Podemos escrever a inversa da transformada como se encontra nas notas fazendo \(L=\frac{1}{2}\), o que faria desaparecer a fracção no início e multiplicando o expoente da exponencial por \(2\pi\).

Repare-se que apenas se \(g\) for somável é que podemos fazer a sua inversa. No entanto, note-se que a inversa poderá não ser \(L^{1}\), aliás, essa é a regra geral. Dá-se agora um exemplo de uma transformada de Fourier.

\(\textbf{Exemplo:}\)

Seja \(f:\quad\mathbb{R}\to\mathbb{R}\), \(f\in L^{1}\) dada por:
\[f(x)= \begin{cases} 1\quad se\quad-1\leq x\leq 1 \\0\quad caso\quad contr\Acute{a}rio \end{cases}\]

Então:

\[\mathcal{F}(f)(x)=\displaystyle\int_{-\infty}^{+\infty}f(x)e^{-ix\xi}\quad dx=\displaystyle\int_{-1}^{1}e^{-ix\xi}\quad dx=-\frac{1}{i\xi}e^{-ix\xi}\Biggr|_{-1}^{1}=\]\[=\frac{e^{-i\xi}-e^{-i\xi}}{i\xi}=\frac{2\sin{\xi}}{\xi}.\]

\subsection{Para \(\mathbb{R}^n\)}
Por extensão, é possível definir a transformada de Fourier para \(\mathbb{R}^n\), desde com as alterações adequadas.

\(\textbf{Definição:}\) Dada uma função \(f:\) \(\mathbb{R}^n\rightarrow\mathbb{C}\) tal que \(f\in L^1(\mathbb{R}^n)\), define-se a sua transformada de Fourier como
\[\mathcal{F}(f)=\hat{f}(\Vec{\xi}) = \displaystyle\int_{\mathbb{R}^n}f(\vec{x})e^{-i\vec{x}\cdot\vec{\xi}}\quad dx\]
A sua inversa é definida por
\[\mathcal{F}^{-1}(g)(\vec{x})=\frac{1}{(2\pi)^{n}} \displaystyle\int_{\mathbb{R}^n}g(\vec{x})e^{-i\vec{x}\cdot\vec{\xi}}\quad d\Vec{\xi}\]

Mais uma vez, apenas se \(g\) for somável é que podemos fazer a sua inversa. No entanto, note-se que a inversa poderá não ser \(L^{1}\).

\section{Propriedades da Transformada de Fourier}

Para finalizar, listo aqui algumas propriedades da transformada de Fourier, dada por \(\mathcal{F}\), válidas \(\mathbb{T}\) e \(\mathbb{R}^{n}\), e a respectiva demonstração:
\begin{itemize}
    \item para \(\mathbb{T}\) e \(\mathbb{R}^{n}\), \(\mathcal{F}\) é um operador linear, isto é
    
    \[\mathcal{F}(\alpha f+\beta g)(\xi)=\alpha\mathcal{F}(f)+\beta\mathcal{F}(g),\quad\alpha,\beta\in\mathbb{R}.\] 
    \textbf{Demonstração:}
    
    Prova-se que \(\mathcal{F}\) é linear em \(\mathbb{R}^{n}\), uma vez que a demonstração para \(f\) em \(\mathbb{T}\) é idêntica. Temos:
    
    \[\mathcal{F}(\alpha f+\beta g)(\vec{\xi})=\displaystyle\int_{\mathbb{R}^{n}}(\alpha f(\vec{x})+\beta g(\vec{x}))e^{-i\vec{x}\cdot\vec{\xi}}\quad dx=\]
    \[=\displaystyle\int_{\mathbb{R}^{n}}\alpha f(\vec{x})e^{-i\vec{x}\cdot\vec{\xi}}+\beta g(\vec{x})e^{-i\vec{x}\cdot\vec{\xi}}\quad dx=\]
    \[=\alpha\displaystyle\int_{\mathbb{R}^{n}}f(\vec{x})e^{-i\vec{x}\cdot\vec{\xi}}\quad dx +\beta\displaystyle\int_{\mathbb{R}^{n}}g(\vec{x}))e^{-i\vec{x}\cdot\vec{\xi}}\quad dx=\]
    \[\alpha\mathcal{F}(f)(\vec{\xi})+\beta\mathcal{F}(g)(\vec{\xi})\]
    
    O que verifica a linearidade da transformada.
    \item para \(\mathbb{T}\) e \(\mathbb{R}^{n}\), \(\mathcal{F}((f)\) é limitada se \(f\in L^{1}\).
    
\textbf{Demonstração:}
Majoramos\cite{maggie2} \(\mathcal{F}(f)=\hat{f}(\xi)\).
\[sup(|\hat{f}(\xi)|)\leq \Biggr|_{}\displaystyle\int_{\mathbb{R}^{n}}f(\vec{x})e^{-i\vec{x}\cdot\vec{\xi}}\quad dx\Biggr|_{}\leq\displaystyle\int_{\mathbb{R}^{n}}|f(\vec{x})| dx = ||f||_{L^{1}}\]
    \item para \(\mathbb{T}\) e \(\mathbb{R}^{n}\), existem funções \(L^{1}\) cuja Transformada de Fourier não está em \(L^{1}\)
    
\textbf{Demonstração:}
Podemos olhar para o exemplo dado na secção \(\mathbf{6.1}\).

A função dada no exemplo é \(L^{1}\), no entanto
\[\mathcal{F}(f)= \frac{2\sin(\xi)}{\xi}\] não é integrável à Lebesgue em \(\mathbb{R}\). É fácil ver isso, uma vez que o integral da função em \([0,+\infty[\) diverge.

Sabemos que \(|\sin{x}|<1\), como tal, o integral do módulo é tal que:
\[\displaystyle\int_{0}^{+\infty}\Biggr| \frac{\sin{x}}{x}\Biggr|\quad dx= \displaystyle\sum_{n=0}^{\infty}\displaystyle\int_{k\pi}^{(k+1)\pi}\Biggr| \frac{\sin{x}}{x}\Biggr|\quad dx\quad >\]
\[>\quad\displaystyle\sum_{n=0}^{\infty}\displaystyle\int_{k\pi}^{(k+1)\pi}\frac{|\sin{x}|}{k\pi}=\displaystyle\sum_{n=0}^{\infty}\frac{1}{k\pi}.\]
É sabido que esta série diverge, logo o integral também diverge e, portanto, \(\frac{\sin{x}}{x}\notin L^{1}(\mathbb{R})\).

\end{itemize}

\printbibliography[title={Referências}]
\end{document}
