\documentclass[11pt,onecolumn]{article}
\usepackage{booktabs, multirow} % for borders and merged ranges
\usepackage{amsmath,amssymb,graphicx,algorithmic,xspace,url}
\usepackage[titlenotnumbered,noend,noline]{algorithm2e}
\usepackage{enumerate}
\usepackage{mathrsfs}
\usepackage{color,soul}
\usepackage{array}
\usepackage{csquotes}
\usepackage{dirtytalk}



\setlength{\pdfpagewidth}{8.5in}
\setlength{\pdfpageheight}{11in}
\setlength{\evensidemargin}{-0.1in} \setlength{\oddsidemargin}{-0.1in}
\setlength{\textwidth}{6.5in} \setlength{\textheight}{9.0in}
\setlength{\topmargin}{-0.6in}

\sloppypar
\begin{document}

\begin{center}
\begin{Large}
ECE 493, Spring 2020, Assignment 1\\
Due: Friday June 19, 11:59pm
\end{Large}
\end{center}
\vspace{.2em}

\vspace{.2em}
\begin{center}
Submit to the UWaterloo Crowdmark site using the link you received via email.
Your answer can be handwritten converted to an electronic file by and scanner or camera; or the answers can be typed up in a word processor or LaTeX and submitted as a pdf.
\end{center}
\vspace{.2em}

\vspace{.2em}


\section{Basics of Probability}

\begin{enumerate}
    \item Let $X$ and $Y$ be two random variables. If $X$ and $Y$ are uncorrelated, does it imply $X$ and $Y$ are independent? If yes, show how, else provide a counterexample. 
    \setlength{\parskip}{6pt}

    Definitions:
    \begin{itemize}
        \item $Cov(X,Y) = E[XY] - E[X]E[Y]$
        \item $Corr(X,Y) = \frac{Cov(X,Y)]}{\sqrt{Var(X)*Var(Y)}}$
    \end{itemize}
    To be uncorrelated means to have a $Covariance$ of 0:
    \begin{equation}
        E[XY] = E[X]E[Y].
    \end{equation}
    \hl{$\therefore$ $X$ and $Y$ are independent.} 
    
    \item A satellite is sending a binary message/code, i.e.\ a sequence of 1s and 0s. Suppose 70\% of the message to be sent is 0s. There is a 80\% chance of a given 0 or 1 being received correctly. Find the probability that 0 was sent if 1 was received. 
    \setlength{\parskip}{6pt}

    Definitions:
    \begin{itemize}
        \item $P(A \vert B) = \frac{P(B \vert A)*P(A)}{P(B \vert A)P(A) + P(B \vert A^c)P(A^c)}$
    \end{itemize}
    Let the event $transmission=0_2$ be represented as $t0_2$ and the event  $received = 1_2$ be represented as $r1_2$.
    \begin{equation}
        P(transmission=0_2 \vert received = 1_2) =  P(t0_2 \vert r1_2)
    \end{equation}
    
    \begin{equation}
        \begin{aligned}
        P(t0_2 \vert r1_2) & =  \frac{P(r1_2 \vert t0_2)P(t0_2)}{P(r1_2 \vert t0_2)P(t0_2) + P(r1_2 \vert t1_2)P(t1_2)} \\ 
        & = \frac{(1-0.8)*0.7}{(1-0.8)*0.7 + 0.8*0.3} \\
        & = 0.37
        \end{aligned}
    \end{equation}
    \hl{$\therefore$ The probability that a $0_2$ was sent given that a $1_2$ was received is $37\%$.}
    \item \textbf{Basic Inference}: The probability of getting a headache is $1/10$, getting the flu is $1/40$, and if you have a flu, there's $1/2$ chance of getting the headache. One day you wake up with a headache and say `Dang, I have a headache. Since 50\% of flues are associated with headaches, I must have a 50-50 chance of coming down with the flu'. Is this line of reasoning correct? Why/Why not?
    \setlength{\parskip}{6pt}

    Definitions:
    \begin{itemize}
        \item $P(A \vert B) = \frac{P(B \vert A)*P(A)}{P(B \vert A)P(A) + P(B \vert A^c)P(A^c)}$
    \end{itemize}

    This line of reasoning is not correct because the problem states that the of getting the a headache given you have the flu is 50\%. The problem \textbf{does not} state that the probability of getting the flu given you have a headache is 50\%. These two statements are not equivalent.

    Let $h$ be the event that you have a headache, and $f$ be the event that you have the flu. The problem describes the following conditional probability:
    \begin{equation}
        \begin{aligned}
        P(f \vert h) & =  \frac{P(h \vert f)*P(f)}{P(h \vert f)P(f) + P(h \vert f^c)P(f^c)} \\
        & =  \frac{0.5*0.025}{0.5*0.025 + 0.1*(1-0.025)} \\ 
        & =  0.11 \\ 
    \end{aligned}
    \end{equation}
    \hl{$\therefore$ There is only an 11\% chance of getting the flu given you have a headache, not 50\%.}

    \item \textbf{Bayesian Learning}: We are interested in forming a belief (hypothesis) about our environment based on what we have observed (evidence). Bayes rule is important when we want to assign a probability to our hypothesis after observing some evidence. Recall Bayes Rule:
    
    \begin{equation*}
        \mathbb{P}(H|e) = \frac{\mathbb{P}(e|H) \mathbb{P}(H)}{\mathbb{P}(e)},
    \end{equation*}
    where $H$ is the hypothesis, $e$ is the evidence, $\mathbb{P}(H)$ is our prior distribution over the different hypotheses, $\mathbb{P}(e|H)$ is the likelihood of the evidence given our hypothesis, $\mathbb{P}(H|e)$ is the posterior distribution of the different hypotheses given the evidence, and $\mathbb{P}(e)$ is the probability of the evidence.
    
    Suppose there are five kinds of bags of marbles: 
    \begin{enumerate}
        \item 10\% are bags of type $h_1$ with 100\% red marbles.
        \item 20\% are bags of type $h_2$ with 75\% red marbles and 25\% green marbles.
        \item 40\% are bags of type $h_3$ with 50\% red marbles and 50\% green marbles.
        \item 20\% are bags of type $h_4$ with 25\% red marbles and 75\% green marbles.
        \item 10\% are bags of type $h_5$ with 100\% green marbles.
    \end{enumerate}
    
    We buy a bag at random, pick 5 marbles from the bag and observe them to be all green. We are interested in knowing what type of bag we bought and what colour the next marble will be. 
    
    Let $e$ be the event that the 5 marbles drawn are green and let $\mathbb{P}(e) = \alpha$. 
    
    \hl{ASSUMPTION: Each marble pick is independent (picking marbles then putting thme back in the bag).}

    \begin{enumerate}
        \item Given that the 5 marbles drawn were green, compute the probability of each type of bag i.e.\ what is the probability of bag type $h_i$ for $i = \{1, 2, 3, 4, 5\}$, given that 5 marbles picked were green?
        \setlength{\parskip}{6pt}
            
        Definitions:
        \begin{itemize}
            \item $P(A \vert B) = \frac{P(B \vert A)*P(A)}{P(B \vert A)P(A) + P(B \vert A^c)P(A^c)}$
        \end{itemize}
        \begin{equation}
            \begin{aligned}
                P(h_i \vert 5g) &= \frac{P(5g \vert h_i)*P(h_i)}{P(5g \vert h_i)*P(h_i) + P(5g \vert h_i^c)P(h_i^c)} \\ 
            \end{aligned}
        \end{equation}
        \begin{equation}
            \begin{aligned}
                P(h_1 \vert 5g) &= \frac{0*0.1}{0*0.1 + 0.25^5*0.2 +  0.5^5*0.4 + 0.75^5*0.2 + 1^5*0.1} \\
                & = 0
            \end{aligned}
        \end{equation}
        \begin{equation}
            \begin{aligned}
                P(h_2 \vert 5g) &= \frac{0.25^5*0.2}{0*0.1 + 0.25^5*0.2 +  0.5^5*0.4 + 0.75^5*0.2 + 1^5*0.1} \\
                & = 0.00122
            \end{aligned}
        \end{equation}

        \begin{equation}
            \begin{aligned}
                P(h_3 \vert 5g) &= \frac{0.5^5*0.4}{0*0.1 + 0.25^5*0.2 +  0.5^5*0.4 + 0.75^5*0.2 + 1^5*0.1} \\
                & = 0.078
            \end{aligned}
        \end{equation}

        \begin{equation}
            \begin{aligned}
                P(h_4 \vert 5g) &= \frac{0.75^5*0.2}{0*0.1 + 0.25^5*0.2 +  0.5^5*0.4 + 0.75^5*0.2 + 1^5*0.1} \\
                & = 0.30
            \end{aligned}
        \end{equation}

        \begin{equation}
            \begin{aligned}
                P(h_5 \vert 5g) &= \frac{1^5*0.1}{0*0.1 + 0.25^5*0.2 +  0.5^5*0.4 + 0.75^5*0.2 + 1^5*0.1} \\
                & = 0.62
            \end{aligned}
        \end{equation}
        
        \item Use the results from part (a) to find the value of $\alpha$ using the axioms of probability.
        \setlength{\parskip}{6pt}

        Definitions:
        \begin{itemize}
            \item $P(A) = P(A \vert B)*P(B) + P(A \vert B^c)*P(B^c)$
        \end{itemize}
        
        \begin{equation}
            \begin{aligned}
                P(e) & = \sum_{i=1}^{i=5}P(e \vert h_i)*P(h_i) \\ 
                    & = 0*0.1 + 0.25^5*0.2 +  0.5^5*0.4 + 0.75^5*0.2 + 1^5*0.1
                    & = 0.16
            \end{aligned}
        \end{equation}
        \hl{$\therefore \alpha = 0.16$}
        \item Let $X$ be a random variable that represents the outcome of the next pick from the bag i.e.\ $X$ can be either green or red. How would you make a prediction about the outcome of $X$, given that the first 5 marbles picked were green? (Give an equation for $\mathbb{P}(X|e)$) Hint: Think about how you can use the concept of joint and marginal distributions with the events of picking different types of bags.
        \setlength{\parskip}{6pt}

        Let $Y$ represent a random variable for the number of green marbles drawn from the bag. Then to solve for 
        \begin{equation}
            p_{X\vert Y}(x \vert e) = \frac{p_{YX}(e,x)}{p_Y(e)} 
        \end{equation}

        \item Use the result from part (c) to compute $\mathbb{P}(X=\text{"green"} |e)$.
        \setlength{\parskip}{6pt}

        Let the event of the 6th marble picked being green be represented as $g$.
        \begin{equation}
            \begin{aligned}
                p_{X\vert Y}(g \vert e) & = \frac{p_{YX}(e, g)}{p_Y(e)} \\
            \end{aligned}
        \end{equation}
        From 4.b, it can be seen that:
        \begin{equation}
            \begin{aligned}
                p_{Y}(e) & = 0.16. 
            \end{aligned}
        \end{equation}
        Following a similar process as 4.a \& 4.b, it can be seen that:
        \begin{equation}
            \begin{aligned}
                p_{X}(g) & = 0*0^1*0.1 + 0.25^1*0.2 +  0.5^1*0.4 + 0.75^1*0.2 + 1^1*0.1 \\
                & = 0.5
            \end{aligned}
        \end{equation}
        Let H represent the bag picked. Due to the assumption stated above, of marble pickings being independent:
        \begin{equation}
            \begin{aligned}
                p_{YX}(e, g) & = \sum_{i=1}^{i=5}p_{YX \vert H=h_i}(e, g) \\
                & = \sum_{i=1}^{i=5}p_{Y \vert H}(e, h_i)*p_{X \vert H}(g, h_i)*P(h_i) \\
                & =  0*0^1*0*0.1 + 0.25^1*0.25^5*0.2 +  0.5^1*0.5^5*0.4 + 0.75^1*0.75^5*0.2 + 1^1*1^5*0.1 \\
                & =  0 + 0.25^6*0.2 +  0.5^6*0.4 + 0.75^6*0.2 + 1^6*0.1 \\
                & = 0.142
            \end{aligned}
        \end{equation}
        Putting it all together:
    \begin{equation}
        \begin{aligned}
            p_{X\vert Y}(g \vert e) & = \frac{p_{XY}(g,e)}{p_Y(e)} \\
            & = \frac{0.142}{0.16} \\
            & = 0.89
        \end{aligned}
    \end{equation}
    \hl{$\therefore$ The probability of picking a green marble after picking 5 green marbles is 89\%.}
    \end{enumerate}
    
\end{enumerate} 
\end{document}