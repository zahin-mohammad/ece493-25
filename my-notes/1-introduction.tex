\documentclass{article}
\usepackage[utf8]{inputenc}
\usepackage{biblatex}
\usepackage{amssymb}
\usepackage{amsmath}
\usepackage{marginnote}

\addbibresource{bib.bib}
\setlength{\parindent}{0em}
\bibliography{bib}


\title{Introduction}
\author{Zahin Mohammad}
\date{May 2020}
\linespread{1.5}
\begin{document}

\maketitle

\section{Probabilistic modelling}

    Simplest model 
    \begin{equation}
        y = \beta^Tx,
    \end{equation}
    where $\beta$ are the parameters, and $x$ is a series of factors that affect output $y$. Can think of $x$ as all the factors that affect the price of a house defined by $y$. This is not ideal because real world has no guarantees.
    \setlength{\parskip}{6pt}
    More useful to know the probability that the price of the house, $y$, is some value given factors of $x$
    \begin{equation}
        P(x,y)
    \end{equation}
\section{The difficulties of probabilistic modelling}
    Probabilities are inherently exponentially-sized objects; the only way in which we can manipulate them is by making simplifying assumptions about their structure.
    \setlength{\parskip}{6pt}

    The Conditional Independence simplifying assumption: given the output y, the input variables are independent. Ex. probability of two English words appearing are independent if the email is spam.
    
    Recall:
    \begin{itemize}
        \item $P(A,B) = P(A|B)P(B)$
    \end{itemize}

    Relating back to Conditional Independence:
    \begin{itemize}
        \item $P(y,x) = P(y,x_1,x_2,...x_n) = P(x_1|y)P(x_2|y),...P(x_n|y)P(y)$ 
        \item $P(y,x) = P(y) \prod_{i=1}{P(x_i|y)}$
    \end{itemize}

    "Each factor $P(x_i|y)$ can be completely described by 4 parameters with 2 degrees of freedom to be exact". What does this mean? $x_i$ has 2 possible values, and $y_i$ has two possible values

\section[]{A bird’s eye overview of the first part of the course}
    
    Graphical model discussion:
    \begin{itemize}
        \item Representation: How to express probability distribution of real-world phenomenon?
        \item Inference: Given probability model, how to find relevant answers.
        \item Learning: Fitting model to a data-set.
    \end{itemize}
    

\printbibliography[title={Referências}]
\end{document}
